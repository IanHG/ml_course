\documentclass[]{article}

%%%%%
% typesetting
%%%%%
\usepackage[OT1]{fontenc}   % 
\usepackage[utf8]{inputenc} % enables use of æ,ø,å
\usepackage[danish,german,english]{babel}
\usepackage{lmodern}


%%%
% verbatim
%%%
\usepackage{verbatim}
\usepackage{fancyvrb} % for \Verb + \Verbatiminput

%%%%
% general packages
%%%%
%\usepackage{etex} % extended tex, for using more than 256 registers for coding

% make danish text with æøå
\newcommand{\textdanish}[1]%
{%
      \foreignlanguage{danish}{#1}%
}

%%%%
% code listing
%%%%
\usepackage{xcolor}
\usepackage{listings}

%%%%
% math
%%%%
\usepackage{amsmath}
\usepackage{amsfonts}
\usepackage{amssymb}

%%%%
% tikz/graphics
%%%%
\usepackage{tikz}
%\usetikzlibrary{arrows.meta}
\usetikzlibrary{arrows}
\usetikzlibrary{calc}
\usetikzlibrary{decorations.pathreplacing}
\usetikzlibrary{decorations.markings}
\usepackage{graphicx}
\usepackage{subcaption} % for subfigure

%%%%%
% for tabulars
%%%%%
\usepackage{multirow}

%%%%
% text colorbox
%%%%
\usepackage[listings,most]{tcolorbox}
\tcbset{
  colback=black,
}

\newcommand{\hiddenhint}[1] %
{ %
\newline
\begin{tcolorbox}{ {\color{black} \emph{Hint:} } {#1} } \end{tcolorbox}
%\begin{tcolorbox}{\color{black} \emph{Hint:} } {#1}\end{tcolorbox}
%\emph{HiddenHint:} {#1}
} %

\newcommand{\hint}[1] %
{ %
\emph{Hint:} {#1}
} %

%%%%%
% spaces
%%%%%
\newcommand{\verticalspace}{\vspace{0.3cm}}
\newcommand{\mediumverticalspace}{\vspace{0.2cm}}
\newcommand{\smallverticalspace}{\vspace{0.1cm}}
\newcommand{\horizontalspace}{\hspace{1cm}}


\usepackage{pgfplots}
\pgfplotsset{compat=1.13}

\usetikzlibrary{external}\tikzexternalize
\usetikzlibrary{shapes}   % for diamond shape
\usetikzlibrary{fit}   % for diamond shape
\usetikzlibrary{shadows.blur}
\usetikzlibrary{decorations.pathreplacing}
\usetikzlibrary{matrix}

\pgfdeclarelayer{bg}    % declare background layer
\pgfsetlayers{bg,main}  % set the order of the layers (main is the standard layer)

% Draw input layer
\newcommand{\drawinputlayer}[1]{
   \foreach \name / \y in {1,...,#1}
   % This is the same as writing \foreach \name / \y in {1/1,2/2,3/3,4/4}
      \node[input neuron, pin={[pin edge={line width=0.3mm}]left:Input \y}] (I-\name) at (0,-\y) {};
}

% Draw hidden layer
\newcommand{\drawhiddenlayer}[3]{
   \foreach \name / \y in {1,...,#1}
      \path[yshift=#3cm]
         node[hidden neuron] (H#2-\name) at (#2 * \layersep, -\y cm) {};
}

% Draw output layer
\newcommand{\drawoutputlayer}[2]{
   \foreach \name / \y in {1,...,#1}
      \path[yshift=-0.5cm]
         node[output neuron, pin={[pin edge={->, line width=0.3mm}]right:Output \y}] (O-\name) at (#2 * \layersep, -\y cm) {};
}

% Connect layers
\newcommand{\connectlayers}[4]{
   \foreach \source in {1,...,#2}
      \foreach \dest in {1,...,#4}
         \path (#1-\source) edge[line width=0.3mm] (#3-\dest);
}

\begin{document}
\def\layersep{2.5cm}
\begin{tikzpicture}
   \tikzstyle{arrow}=[->,draw=black!70, node distance=\layersep]
   \tikzstyle{every pin edge}=[<-,shorten <=1pt]
   \tikzstyle{neuron}=[circle,thick,draw=black,minimum size=17pt,inner sep=0pt]
   \tikzstyle{input}=[text=blue,minimum size=22pt];
   \tikzstyle{weight}=[neuron, minimum size=22pt];
   \tikzstyle{sum}=[neuron, minimum size=30pt];
   \tikzstyle{activation}=[neuron, rectangle, minimum size=30pt];
   \tikzstyle{annot} = [text width=20em, text centered];
   \tikzstyle{layerbg} = [rectangle, rounded corners];
   
   \node[] at (0,0) (W) {
      \(\boldsymbol{W} = 
      \begin{bmatrix}
         1 & 1 \\
         1 & 1 \\
      \end{bmatrix}\)

   };
   \node[xshift=1.2cm] at (W.east) (c) {
      \(\boldsymbol{c} = 
      \begin{bmatrix}
         0 \\
         -1 \\
      \end{bmatrix}\)
   };
   \node[xshift=1.2cm] at (c.east) (w) {
      \(\boldsymbol{w} =
      \begin{bmatrix}
         1  \\
         -2 \\
      \end{bmatrix}\)
   };
   \node[xshift=1.2cm] at (w.east) (b) {
      \( b = 0 \)
   };
   \node[xshift=+0.75cm, yshift=+0.5cm] at (c.north) (f) {
      \(
      f\left( \boldsymbol{x}; \boldsymbol{W}, \boldsymbol{c}, \boldsymbol{w}, b \right) = \boldsymbol{w}^T \max \left(0, \boldsymbol{W}^T \boldsymbol{x} + \boldsymbol{c} \right) + b
      \)
   };

   \begin{pgfonlayer}{bg}
      \node[layerbg,fit={(W) (c) (w) (b) (f)},very thick, draw=gray,inner sep=0.25cm] (nbg) {};
   \end{pgfonlayer}
   \node[annot, centered, font=\large, yshift=0.5cm, xshift=-0.2cm, text=gray] (pc) at (nbg.north) {The complete XOR model};
\end{tikzpicture}

\begin{tikzpicture}
   \tikzstyle{arrow}=[->,draw=black!70, node distance=\layersep]
   \tikzstyle{every pin edge}=[<-,shorten <=1pt]
   \tikzstyle{neuron}=[circle,thick,draw=black,minimum size=17pt,inner sep=0pt]
   \tikzstyle{input}=[text=blue,minimum size=22pt];
   \tikzstyle{weight}=[neuron, minimum size=22pt];
   \tikzstyle{sum}=[neuron, minimum size=30pt];
   \tikzstyle{activation}=[neuron, rectangle, minimum size=30pt];
   \tikzstyle{annot} = [text width=20em, text centered];
   \tikzstyle{layerbg} = [rectangle, rounded corners];
   
   % Matrices
   \node[] at (0,0) (X) {
      \(
      \begin{bmatrix}
         0 & 0 \\
         0 & 1 \\
         1 & 0 \\
         1 & 1 \\
      \end{bmatrix}
      \)

   };
   \node[xshift=-0.2cm] at (X.west) {
      \( X : \)
   };
   \node[] at (\layersep,0) (XW) {
      \(
      \begin{bmatrix}
         0 & 0 \\
         1 & 1 \\
         1 & 1 \\
         2 & 2 \\
      \end{bmatrix}
      \)
   };
   \node[] at (2 * \layersep,0) (c) {
      \(
      \begin{bmatrix}
         0 & -1 \\
         1 & 0 \\
         1 & 0 \\
         2 & 1 \\
      \end{bmatrix}
      \)
   };
   \node[] at (3 * \layersep,0) (rect) {
      \(
      \begin{bmatrix}
         0 & 0 \\
         1 & 0 \\
         1 & 0 \\
         2 & 1 \\
      \end{bmatrix}
      \)
   };
   \node[] at (4 * \layersep,0) (res) {
      \(
      \begin{bmatrix}
         0 \\
         1 \\
         1 \\
         0 \\
      \end{bmatrix}
      \)
   };

   % Arrows
   \path (X) edge[arrow,line width=0.3mm] node [above,midway,font=\scriptsize] {\(\boldsymbol{W} \)} (XW);
   \path (XW) edge[arrow,line width=0.3mm] node [above,midway,font=\scriptsize] {\(+ \boldsymbol{c} \)} (c);
   \path (c) edge[arrow,line width=0.3mm] node [above,midway,font=\scriptsize] {\textbf{ReLU}} (rect);
   \path (rect) edge[arrow,line width=0.3mm] node [above,midway,font=\scriptsize] {\(\boldsymbol{w}^T\)} (res);

   %\begin{pgfonlayer}{bg}
   %   \node[layerbg,fit={(W) (c) (w) (f)},very thick, draw=gray,inner sep=0.25cm] (nbg) {};
   %\end{pgfonlayer}
   %\node[annot, centered, font=\large, yshift=0.5cm, xshift=-0.2cm, text=gray] (pc) at (nbg.north) {The complete XOR model};
\end{tikzpicture}

   

\end{document}
