%%%%
%
% includes needed
%
%%%%
\usepackage{color}
\usepackage{xcolor}
\usepackage[margin=10pt]{caption}
\usepackage{listings}
\usepackage{fancyvrb} % for \Verb

%%%%
%
% General settings
%
%%%%
% color setup
\definecolor{listinggray}{gray}{0.98}
%\colorlet{listinggray}{gray!40}
\definecolor{lbcolor}{rgb}{0.9,0.9,0.9}

\lstset{
   frame=tb
 , frameround=ftft
 , numbers=left      % put code line numbers on left
 , numberstyle=\tiny\color[rgb]{0.205, 0.142, 0.73}% size of line numbers
 , numbersep=5pt
 , xleftmargin=\parindent % set left margin to paragraph indention
 , captionpos=tl
 %, backgroundcolor=\color{lbcolor}
 , backgroundcolor=\color{listinggray}
 , keywordstyle=\color[rgb]{0,0,1}
 , commentstyle=\color{violet}
 , stringstyle=\color{red}
 , basicstyle=\ttfamily\lst@ifdisplaystyle\scriptsize\fi
}

% set code size
%\def{\codesize}{\lst@ifdisplaystyle\scriptsize\fi}

\newcommand{\inlinecode}[2][cpp]{\lstinline[style=custom#1,prebreak=]{#2}}
\newcommand{\includecode}[3][cpp]{\lstinputlisting[caption={#2 \hfill
\texttt{\scriptsize (\lstname)}},style=custom#1]{#3}}

% caption setup
\DeclareCaptionLabelFormat{advancedlabelformat}{[ #1 ]}
%\DeclareCaptionFormat{advancedformat}{\colorbox{cmyk}{0.43, 0.35, 0.35,0.01}{\parbox{\textwidth}{#1 #3}}}
\DeclareCaptionFormat{advancedformat}{#1 #3}
\captionsetup[lstlisting]{
   format=advancedformat
 , labelformat=advancedlabelformat
}


%%%%
%
% Cpp enviroment
%
%%%%
\newcommand{\cppname}{{C{}\Verb|++|}} % name of cpp listings

% cpp style
\lstdefinestyle{customcpp}{
   language=[ISO]C++
 %, keywordstyle=\color{red!20!darkgreen}\bfseries
 %, keywordstyle=\color{yellow!60!green}\bf
 %, commentstyle=\color{blue}
 %, stringstyle=\ttfamily\color{red!50!brown}
 , showstringspaces=false
 , escapechar=£
 %, basicstyle=\ttfamily\codesize
 , flexiblecolumns=true
 , keywords=[2]{final,override,nullptr,uchar,size_t}
 %, literate=%
 %     *{0}{{{\color{red!20!violet}0}}}1
 %      {1}{{{\color{red!20!violet}1}}}1
 %      {2}{{{\color{red!20!violet}2}}}1
 %      {3}{{{\color{red!20!violet}3}}}1
 %      {4}{{{\color{red!20!violet}4}}}1
 %      {5}{{{\color{red!20!violet}5}}}1
 %      {6}{{{\color{red!20!violet}6}}}1
 %      {7}{{{\color{red!20!violet}7}}}1
 %      {8}{{{\color{red!20!violet}8}}}1
 %      {9}{{{\color{red!20!violet}9}}}1
}

% cpp environment
\lstnewenvironment{cpp}[1][]
{ 
   \renewcommand\lstlistingname{\cppname{}}
   \lstset{style=customcpp,caption={#1 \hfill \ }}
}
{ }

% inline cpp
\newcommand{\icpp}[1]{
   \inlinecode[cpp]{#1}
}

% include cpp file
\newcommand{\includecpp}[2][]{
   \renewcommand\lstlistingname{\cppname{}}
   \includecode[cpp]{#1}{#2}
}


%%%%
%
% Fortran
%
%%%%
\newcommand{\fortranname}{FORTRAN}

\lstdefinestyle{customfortran}{
   language=fortran
 , basicstyle=\ttfamily\codesize
}

\lstnewenvironment{fortran}[1]
{ 
   \renewcomman\lstlistingname{\fortranname{}}
   \lstset{style=customfortran,caption={#1 \hfill \ }} 
}
{ }

\newcommand{\ifortran}[1]{\inlinecode[fortran]{#1}}



%%%%%
%
% Assembly
%
%%%%%
\newcommand{\asmname}{ASM}

\lstdefinestyle{customasm}{
   language=[x86masm]Assembler
 %, basicstyle=\codesize\ttfamily
 , commentstyle=\itshape\color{purple!40!black}
}

\lstnewenvironment{asm}[1]
{
   \renewcommand\lstlistingname{\asmname{}}
   \lstset{style=customasm,caption={#1 \hfill \ }} 
}
{ }

\newcommand{\iasm}[1]{\inlinecode[asm]{#1}}

%%%%
% bash
%%%%
\newcommand{\bashname}{Bash}

\lstdefinestyle{custombash}{
   language=bash
 %, basicstyle=\codesize\ttfamily
}

\lstnewenvironment{bash}[1]
{
   \renewcommand\lstlistingname{\bashname{}}
   \lstset{style=custombash,caption={#1 \hfill \ }} 
}
{ }

\newcommand{\includebash}[2][]{
   \renewcommand\lstlistingname{\bashname{}}
   \includecode[bash]{#1}{#2}
}

%%%%
% terminal
%%%%%
\lstdefinestyle{customterminal}{
   language=bash
 %, backgroundcolor=\color{white}
 , backgroundcolor=\color{listinggray}
 , keywordstyle=\color{black}
 , commentstyle=\color{black}
 , stringstyle=\color{black}
 , numberstyle=\color{black}
 , frame=
 , numbers=none  % put code line numbers on left
 %, basicstyle=\codesize\ttfamily
}

%\newenvcommand{terminal}{\command}[1]
%{
%$ #1
%}

\lstnewenvironment{terminal}[1][]
{
 % \checkenvcommand
  \renewcommand\lstlistingname{Terminal}
  \lstset{style=customterminal,caption={#1 \hfill \ }} 
}
{ }

\newcommand{\iterminal}[1]{\inlinecode[terminal]{#1}}
